\documentclass[12pt,a4paper]{article}
\usepackage[utf8]{inputenc}
\usepackage{amsmath}
\usepackage{amsfonts}
\usepackage{amssymb}
\usepackage{color}
\usepackage{graphicx}


\usepackage[left=2cm,right=2cm,top=2cm,bottom=2cm]{geometry}
\setlength{\parindent}{0pt}


\begin{document}

%%%%%%%%%%%%%%%%%%%%%%%%%%%%%%%%%%%%%%%%%%%%%%%%%%%%%%
%
%  Header
%
%%%%%%%%%%%%%%%%%%%%%%%%%%%%%%%%%%%%%%%%%%%%%%%%%%%%%%
\begin{center}
CSE 373    \hspace{0.4 cm}  
{\bf Data Structures and Algorithms }
  \hspace{0.4 cm}   Summer 2018
\end{center} 
\vspace{-7 mm}
\noindent \hrulefill
\vspace{3 mm}


%%%%%%%%%%%%%%%%%%%%%%%%%%%%%%%%%%%%%%%%%%%%%%%%%%%%%%
%
%  Assignment number and due date
%
%%%%%%%%%%%%%%%%%%%%%%%%%%%%%%%%%%%%%%%%%%%%%%%%%%%%%%

\begin{center}
{\bf \Large Homework 3}

Due Monday, July 23rd at 11:59pm
\end{center}


{\bf Name:} Zachary McNulty

{\bf Student ID:} 1636402\\

%%%%%%%%%%%%%%%%%%%%%%%%%%%%%%%%%%%%%%%%%%%%%%%%%%%%%%
%
%  Problem 1: 
%
%%%%%%%%%%%%%%%%%%%%%%%%%%%%%%%%%%%%%%%%%%%%%%%%%%%%%%

{\bf\large Problem 2: Solving Recurrences}\\


(i)\\ 
Master Theorem Parameters: $d = 1, a = 7, b = 2, c = 2$ \\
Thus, we can see that $log_b (a) = log_2 (7) > 2 = c$. As such, $log_b (a) > c$ and thus: \\ 
$ T(n) \in \Theta(n^{log_b (a)}) = 	\Theta(n^{log_2 (7)})$ \\

(ii) \\
Master Theorem Parameters: $d = 1, a = 4, b = 2, c = 2$ \\
Thus, we can see that $log_b (a) = log_2 (4) = 2 = c$. As such, $log_b (a) = c$ and thus: \\
$T(n) \in \Theta(n^c log (n)) = \Theta(n^2 log (n))$ \\

(iii) \\
Master Theorem Parameters: $d = 1, a = 2, b = 2, c = \frac{1}{2}$ \\
Thus, we can see that $log_b (a) = log_2 (2) = 1 > 1/2 = c$. As such, $log_b (a) > c$ and thus: \\ 
$T(n) \in \Theta(n^{log_b (a)}) = \Theta(n^{log_2 (2)}) = \Theta(n)$\\

(iv) \\
Master Theorem Parameters: $ d = 1, a = 4, b = 2, c = 3$\\
Thus, we can see that $log_b (a) = log_2 (4) = 2 < 3 = c$ As such, $log_b (a) < c$ and thus: \\
$T(n) \in \Theta(n^c) = \Theta(n^3)$ \\

(v) \\
Master Theorem Parameters: $ d  = 1, a = 3, b = 2, c = 1$\\
Thus, we can see that $log_b (a) = log_2 (3) > 1 = c$. As such, $log_b (a) > c$ and thus: \\
$T(n) \in \Theta(n^{log_b (a)}) = \Theta(n^{log_2 (3)})$ \\

\vskip 10in



b)\\

$T(n)$ \\
$= T(n^{0.5}) + T(n^{0.5}) + log (n)$\\
$= (T( (n^{0.5})^{0.5} ) + T( (n^{0.5})^{0.5} ) + log (n^{0.5})) + T( (n^{0.5})^{0.5} ) + T( (n^{0.5})^{0.5} ) + log (n^{0.5})) + log (n)$\\

\begin{tabular}{|c | c|}
\hline
$i = 0$:  & $log(n)$ \\
\hline
$i = 1$: & $log(n^{0.5})$   $log(n^{0.5})$\\
\hline
$i = 2$:  & $log( (n^{0.5})^{0.5}) )$   $log((n^{0.5})^{0.5}))$ $log((n^{0.5})^{0.5}))$   $log((n^{0.5})^{0.5}))$ \\
\hline
\end{tabular}

\vskip 1in

\begin{tabular}{|c|p{4 in}|}
\hline
\textbf{$\#$ nodes at level i: }  &$2^i$ \\
\hline
\textbf{input size at level i:} & $n^ {0.5^i}$\\
\hline
\textbf{work per node at level i:} & $log (size) = log(n^{0.5^i}) = 0.5^i * log(n)$\\
\hline
\textbf{total work at level i:}  & $work_i * nodecount_i = (2^i)( (0.5^i)log(n)) = log(n)$.\\
\hline
\textbf{level base case:} &  $n^{0.5^i} = 2 \rightarrow 0.5^i * log_2(n) = log_2(2) = 1$ \newline
$ \rightarrow  log_2(0.5^i * log_2(n)) = log_2 (1) = 0$\newline
$\rightarrow   i*log_2(0.5) = - log_2(log_2(n))$ \newline
$\rightarrow i = log_2(log_2(n))$ \\
\hline
\textbf{number nodes base case:} & $2^{i} = 2^{log_2(log_2(n))} = log_2(n)$ \\
\hline
\textbf{expression for recursive work:} & $\sum_{i = 0}^{log_2(log_2(n)) - 1} log(n)$ \\
\hline
\textbf{expression for non-recursive work:} & $log_2(n) * 1$\\
\hline
\textbf{closed form for total work: } & $recursive: log(n) + log(n) + ... + log(n)$ \newline
$= log(n) * log_2(log_2(n))$\newline
total = $ log(n) * log_2(log_2(n)) + log_2(n)$ \\
\hline
\textbf{simpliest big $\Theta$ for total work: } & $\Theta(log(n)*log(log(n)))$\\
\hline

\end{tabular}


\end{document}










